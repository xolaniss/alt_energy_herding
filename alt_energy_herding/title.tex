\title{Do investors in clean energy ETFs herd?}



\author{ 
Vassilios Babalos\footnote{Corresponding Author. Department of Accounting and Finance, University of Peloponnese, Kalamata,  Greece; Email: v.babalos@uop.gr.} \,\,
Xolani Sibande\footnote{Department of Economics, University of Pretoria, Pretoria, South Africa; Email: xolaniss@gmail.com. Economic Research Department, South African Reserve Bank, Pretoria, South Africa.} \,\, 
Elie Bouri\footnote{Adnan Kassar School of Business, Lebanese American University, Lebanon; Email: elie.elbouri@lau.edu.lb} \,\,
Rangan Gupta\footnote{Department of Economics, University of Pretoria, Pretoria, South Africa; Email: rangan.gupta@up.ac.za.} 
}
\date{\today}
\maketitle

\begin{abstract}

This study offers novel and valuable insights into herding behaviour in US clean energy ETFs between 2016 and 2023. 
The baseline herding tests by \cite{christie1995} and \cite{chang2000} revealed significant herding behaviour in this market. 
This evidence was supported by asymmetric and time-varying herding tests. That is, investors herd both in bear and bullish markets, and periodically. 
In addition, we found that climate risks (both physical and transition) reduced the probability of herding in US clean energy ETFs, 
indicating that an increase in climate-related risk encouraged efficient or climate-hedging behaviour by investors. 
Therefore, the results suggest that climate-related uncertainty did not drive herding behaviour in this market. The results suggest that investors are appropriately identifying opportunities that mitigate climate risk,
thereby reducing the probability of herding-driven system risk.

\end{abstract}

\noindent\textbf{Keywords}: Herding Behaviour, Climate Change, Energy
\\
\textbf{JEL Codes}: G14, Q54, P18
\newpage