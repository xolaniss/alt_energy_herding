\title{Do investors in clean energy ETFs herd? The role of climate risks}



\author{ 
Vassilios Babalos\footnote{Corresponding Author. Department of Accounting and Finance, University of Peloponnese, Kalamata,  Greece; Email: v.babalos@uop.gr.} \,\,
Xolani Sibande\footnote{Department of Economics, University of Pretoria, Pretoria, South Africa; Email: xolaniss@gmail.com. Economic Research Department, South African Reserve Bank, Pretoria, South Africa.} \,\, 
Elie Bouri\footnote{Adnan Kassar School of Business, Lebanese American University, Lebanon; Email: elie.elbouri@lau.edu.lb} \,\,
Rangan Gupta\footnote{Department of Economics, University of Pretoria, Pretoria, South Africa; Email: rangan.gupta@up.ac.za.} 
}
\date{\today}
\maketitle

\begin{abstract}

\textbf{Purpose} -- This study investigates herding behaviour in US clean energy Exchange-traded funds (ETFs) and examines the role of climate risks in influencing such behaviour over the period from May 1, 2016 to June 19, 2024.

\textbf{Design/methodology/approach} -- We employ a baseline herding model and extend it to examine asymmetric effects across market conditions. The analysis incorporates time-varying herding measures and examines the impact of both transitional and physical climate risks on herding probability using regression techniques.

\textbf{Findings} -- The baseline model reveals significant herding behaviour in clean energy ETFs. The extended model indicates that herding is present in both down and up markets, with a stronger effect in down markets, suggesting asymmetry. Herding is also found to be time-varying. Notably, high levels of transitional climate risk reduce the probability of herding in clean energy ETFs, whereas physical climate risk does not exert any significant impact on herding probability.

\textbf{Research limitations/implications} -- The study focuses specifically on US clean energy ETFs over a defined period, which may limit generalizability to other markets or asset classes. The findings provide insights into the behavioral dynamics of sustainable investment markets during periods of varying climate risk.

\textbf{Practical implications} -- The results suggest that high levels of transitional climate risk encourage market efficiency in clean energy ETFs and promote climate-hedging behaviour by investors. This has important implications for portfolio managers and policymakers in understanding market dynamics in sustainable finance.

\textbf{Originality/value} -- This study provides novel empirical evidence on the relationship between climate risks and herding behaviour in clean energy ETFs, contributing to the growing literature on behavioral finance in sustainable investment markets.

\end{abstract}

\noindent\textbf{Keywords}: Herding Behaviour, Climate Change, Clean Energy
\\
\textbf{JEL Codes}: G14, Q54, P18
\newpage