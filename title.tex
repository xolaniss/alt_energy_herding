\title{Do investors in clean energy ETFs herd? The role of climate risks}



\author{ 
Vassilios Babalos\footnote{Corresponding Author. Department of Accounting and Finance, University of Peloponnese, Kalamata,  Greece; Email: v.babalos@uop.gr.} \,\,
Xolani Sibande\footnote{Department of Economics, University of Pretoria, Pretoria, South Africa; Email: xolaniss@gmail.com. Economic Research Department, South African Reserve Bank, Pretoria, South Africa.} \,\, 
Elie Bouri\footnote{Adnan Kassar School of Business, Lebanese American University, Lebanon; Email: elie.elbouri@lau.edu.lb} \,\,
Rangan Gupta\footnote{Department of Economics, University of Pretoria, Pretoria, South Africa; Email: rangan.gupta@up.ac.za.} 
}
\date{\today}
\maketitle

\begin{abstract}

This study offers new insights into the herding behaviour in US clean energy Exchange-traded funds (ETFs) over the period from May 1, 2016 to June 19, 2024. 
The baseline model shows significant herding. An extended mode indicates that herding is present in both down and up markets, with a stronger effect in the down market, suggesting an asymmetry. 
Herding is also found to be time-varying. Further analysis shows that the transition climate risk, particularly its high level, reduces the probability of herding in clean energy ETFs, whereas physical climate risk does not exert any significant impact on the probability of herding. 
Thus, large levels of transition climate risk seem to encourage market efficiency in clean energy ETFs  and climate-hedging behavior by investors.

\end{abstract}

\noindent\textbf{Keywords}: Herding Behaviour, Climate Change, Clean Energy
\\
\textbf{JEL Codes}: G14, Q54, P18
\newpage